\chapter{Giới thiệu đề tài}
\section{Mở đầu}
Ngày nay sự phát triển không ngừng của công nghệ thông tin và truyền thông thông tin, điện thoại di động đã trở thành vật dụng không thể thiếu của hầu như tất cả mọi người.

Việc tìm kiếm công việc giờ đây không phải làm một cách thủ công như ngày xưa nữa. Không cần những thông tin đa chiều từ nhiều nguồn khác nhau hay phải xe giấy, đánh dấu những tờ báo tuyển việc. Giờ đây, chỉ với một chiếc điện thoại, mọi người đều có thể tìm cho mình nhiều công việc với những lĩnh vực khác nhau. Thế nhưng, với thực trạng sinh viên vừa tốt nghiệp bị thất nghiệp ngày càng tăng. CV là một thứ không thể thiếu khi một ứng viên ứng tuyển vào công ty. CV chính là một chiếc cầu nối giúp kết nối ứng viên với các nhà tuyển dụng. Với một CV chỉnh chu, hoàn thiện, nó sẽ truyền tải đầy đủ thông tin của ứng viên đến với nhà tuyển dụng.

Đặc biệt với những ứng viên là những sinh viên vừa mới ra trường, việc CV thiếu sót là một điều không thể tránh khỏi. Vì vậy, trang web hướng tới mục tiêu giúp đỡ các sinh viên cách viết một bản CV hoàn chỉnh một cách nhanh chóng. Đồng thời, đưa ra những lời khuyên cho sinh viên chỉnh sửa CV mình sao cho chuyên nghiệp và dễ dàng tạo ấn tượng cho các nhà tuyển dụng. Trang web sẽ cung cấp các mẫu CV chuyên nghiệp, các công cụ hỗ trợ tự động gợi ý nội dung dựa trên vị trí ứng tuyển và cho phép người dùng tuỳ chỉnh thông tin theo phong cách cá nhân. Hơn nữa, hệ thống cũng tích hợp chức năng tìm kiếm việc làm dựa trên nhiều tiêu chí khác nhau như: địa điểm, vị trí công việc, mức lương, kinh nghiệm. Ngoài ra, người dùng có thể chia sẻ CV đã được hoàn thiện của mình để các nhà tuyển dụng có thể tìm đến mình và cũng như chia sẻ kinh nghiệm CV của bản thân mình cho những người đến sau. 

Với tính năng thân thiện, dễ sử dụng, và tối ưu hóa trên đa nền tảng, trang web không chỉ giúp người dùng nhanh chóng có được CV ấn tượng mà còn tạo ra sự kết nối hiệu quả với nhà tuyển dụng. Đề tài không chỉ có giá trị thực tiễn cao, mà còn góp phần nâng cao trải nghiệm tìm việc, thúc đẩy khả năng ứng dụng công nghệ vào quá trình tìm kiếm cơ hội nghề nghiệp, đáp ứng nhu cầu cấp thiết của thị trường lao động hiện nay.

\section{Mục tiêu của đề tài}

\begin{itemize}
    \item Trang web này sẽ cung cấp cho người dùng những bản CV mẫu chất lượng và cũng như công cụ viết CV chuyên nghiệp. Điều này sẽ giúp người dùng tạo ra các bản CV chuyên nghiệp và dễ dàng tuỳ chỉnh theo phong cách cá nhân sao cho phù hợp với ngành nghề của mình.
    \item Tích hợp với chức năng tìm kiếm việc làm dựa trên nhiều tiêu chí khác nhau. Đồng thời, cung cấp công cụ lọc và sắp xếp kết quả tìm kiếm để người dùng dễ dàng tìm thấy cơ hội việc làm phù hợp với bản thân mình.
    \item Cho phép người dùng chia sẻ CV đã hoàn thiện của mình để các nhà tuyển dụng tìm đến và liên hệ.
    \item Các công ty trong hệ thống có thể đăng những bài đăng tìm kiếm nhân tài cho công ty mình hoặc sẽ tìm kiếm nhân sự cho công ty thông qua sàng lọc những bản CV của những người dùng đã chia sẻ CV lên trang web và liên hệ đến họ.
    \item Cung cấp những thông tin, tài liệu hướng dẫn người dùng cách viết CV và cách phỏng vấn hiệu quả.
\end{itemize}

\section{Bố cục của luận văn}

Đồ án này của tôi bao gồm 4 chương và được chia thành các phần chi tiết sau:

\subparagraph{Chương 1: Giới thiệu đề tài}
\begin{itemize}
    \item Giới thiệu đồ án nghiên cứu, xác định tình trạng vấn đề và những động lực để thực hiện dự án.
    \item Mô tả và nêu những mục tiêu có thể đạt được khi thực hiện dự án.
\end{itemize}

\subparagraph{Chương 2: Giới thiệu các công nghệ đã sử dụng}
\begin{itemize}
    \item Giới thiệu tổng quan về các công nghệ đã sử dụng trong dự án.
    \item Phân tích các mặt lợi ích của các công nghệ đang sử dụng so với những công nghệ khác.
    \item Diễn giải lý do tại sao lại chọn những công nghệ này.
\end{itemize}

\subparagraph{Chương 3: Phân tích hệ thống và thiết kế trang web:}
\begin{itemize}
    \item Tìm hiểu, phân tích và thiết kế các lược đồ phác thảo về cơ sở dữ liệu và cách vận hành của hệ thống.
\end{itemize}

 \subparagraph{Chương 4: Thiết kế giao diện người dùng}:
 \begin{itemize}
     \item Mô tả và thiết kế những trang khác nhau của giao diện của người dùng.
 \end{itemize}
 
 \subparagraph{Chương 5: Tóm tắt và kế hoạch tương lai}
 \begin{itemize}
 	\item Mô tả lịch trình hoàn thành Đồ án chuyên ngành của kỳ này.
 	\item Hoạch định trước kế hoạch cho giai đoạn tiếp theo của Đồ án tốt nghiệp.
 \end{itemize}